\section{Durchführung}
\label{sec:Durchführung}
\subsection{Aufbau}
Zur Untersuchung der Photoeffekts wird zurnächst eine Photonenquelle benötigt.
Die Photonenquelle, welche für den Versuch genutzt wird ist eine Quecksilberdampflampe.
Das Licht der Lampe trifft zunächst auf eine Kondensorlinse, danach auf eine Spaltblende und darauf auf eine Abbildungslinse.
Durch die Kondensorlinse wird das Licht zunächst gebündelt und die Abbildungslinse entwirft ein Bild der Spaltblende.
Danach trifft das Licht auf einen Geradsichtsprisma wodurch die Spektrallinien des Licht räumlich getrennt werden.
Diese Trennung wird gebraucht, da nur Licht einer bestimmten Spektrallinie in die Photoelektrode treffen soll.

\subsection{Photozelle}
Die Photozelle wird von einem Schutzgehäuse umgeben, welches nur einen kleinen Einlassspalt aufweist.
In dem Gehäause befindet sich eine evakuierter Glaskörper, der zwei Elektroden enthält.
Die Photokathode besteht aus einer aufgedampften Metall- oder Legierungsschicht.
Auf diese trifft das Licht, welches durch den Spalt fällt.
Die Anode bildet ein kreisförmiger Drahtring, dieser verläuft in einigen Millimeter Abstand parallel zur Kathode.
Zwischen Kathode und Anode kann eine Spannung angelegt werden.

\subsection{experimenteller Ablauf}
Zunächst muss die Hg-Lapme eingeschaltet werden.
Es dauert einige Zeit bis das Licht die gewünschte Intensität erreicht.
Nachdem die Intensität der Lampe ausreichend ist, werden die Spektrallinien der Lampe sichtbar.
Da es leichter ist die Spektrallinie zu finden, die eine starke Intensität aufweisen, werden in dem Versuch nur solche gemessen.
Die Spektrallinie mit entsprechender Intensität und höchster Wellenlänge ist gelb mit $\lambda_{gelb} = 577,579 \si{\nano\meter}$.
Die Photozelle wird so ausgerichtet, das nur das gelbe Licht in den Spalt trifft.
Nun wird die Spannung zwischen Photokathode und Anode auf den Wert gebracht, bei dem gerade kein Strom zwischen diesen messbar ist.
Diese Spannung wird notiert und die Spannung wird um $2\si{\V}$ erhöht.
Nun ist ein Strom messbar, die Stärke der Stroms wird nun in Abhängigkeit zur Spannung notiert.
Die Spannung wird wieder um $2\si{\V}$ erhöht und die beiden Werte werden aufgenommen, dieser Prozess wird bis zu einer Spannung von $20\si{\V}$ wiederholt.


Nachdem die gewünschten Messwerte der gelben Spektrallinie aufgenommen wurden, wird die Photozelle erneut ausgerichtet.
Die zu messende Wellenlänge des Spektrums entspricht nun dem grünen Licht mit $\lambda_{grün} = 546 \si{\nano\meter}$.
Bei dieser Messung wird genauso wie bei der Messung des gelben Spektrums vorgegangen.
Zuletzt wird die Phtotozelle auf das violette Licht $\lambda_{viol} = 434 \si{\nano\meter}$ ausgerichtet.
Auch bei dieser Messung wird genauso vorgegangen wie zuvor bei dem gelben und grünen Licht.
