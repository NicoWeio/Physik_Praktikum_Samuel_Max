\section{Diskussion}
\label{sec:Diskussion}
Auffälig ist, dass zwischen der Wurzel des gemessenen Photostroms $\sqrt{I}$ und der angelegten Spannung $U$ für alle Farbspektren kein linearer Zusammenhang besteht (siehe Abb. \ref{fig:gelb}, \ref{fig:gruen}, \ref{fig:violett}).
Grund ist ein zu großes Messintervall und damit eine zu hohe Beschleunigungsspannung.
\\
Der gemessene Photostrom nähert sich bei wachsender Spannung einem Grenzwert an und stellt somit keinen linearen Zusammenhang dar.
\\
Allgemein ist der Versuch stark Fehleranfällig und die Messwerte können durch kleinste äußere Lichteinflüsse (Öffnen der Tür, Verschieben der Lampe) beeinflusst werden.
Zudem können sich die Linsen und dadurch die Spektrallinien durch Erschütterungen des Tisches verschieben.
\\
In Tab. \ref{tab:vergleich} werden die experimentell bestimmten Verhältnisse $\frac{h}{e_0}$ mit dem Theoriewert \cite{konstanten} verglichen.
\begin{table}
    \centering
    \begin{tabular}{c|cc}
    \toprule
    Messung & $\frac{h}{e_0} \,/\, \si{\frac{\joule}{\ampere}}$ & $\text{Abw.} \,/\, \si{\percent}$ \\
    \midrule
    Theoriewert & $\SI{-4.14e-15}{}$ & - \\
    Ausgleichsrechnung & $\SI{2.3(6)e-14}{}$ & $\SI{660}{}$ \\
    1. Messwert & $\SI{-3.05(33)e-15}{}$ & $\SI{26}{}$ \\
    \bottomrule
    \end{tabular}
    \caption{Das zu bestimmende Verhältnis $\frac{h}{e_0}$ im Vergleich zum Literaturwert \cite{konstanten}.}
    \label{tab:vergleich}
\end{table}
\\
Das Verhältnis, welches über die lineare Ausgleichsrechnung bestimmt wird, hat eine hohe Abweichung zum Theoriewert.
Die Grenzspannungen $U_g$, die direkt gemessen wurden (1. Messwert) können zur Bestimmung des Verhältnissen $\frac{h}{e_0}$ verwendet werden.
Die Abweichung zum Literaturwert liegt bei $\SI{26}{\percent}$.
\\
Um geeignete Messwerte zu erhalten, muss bei einer niedrigen Brems- bzw. Beschleunigungsspannung der Photostrom gemessen werden.