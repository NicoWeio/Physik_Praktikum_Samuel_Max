\section{Theorie}
\label{sec:Theorie}

\subsection{Grundlagen}
Magnetfelder werden durch die Bewegung elektrischer Ladungen erzeugt.
Die magnetische Feldstärke $\vec{H}$ beschreibt die Richtung und Stärke des Magnetfelds.
Der Verlauf des Magnetfeldes kann durch Feldlinien dargestellt werden.
Diese bilden sogenannte Schleifen und sind immer geschlossen.\\
Häufig besitzen Atome allein durch ihre Elektronenanordnung und -bewegung ein dauerhaftes magnetisches Moment.
In dem Fall, dass die magnetischen Momente statistisch verteilt sind, kann die magnetische Flussdichte $\vec{B}$
\begin{equation}
    \vec{B} = \mu \cdot \vec{H}
\end{equation}
verwendet werden.\\
Dabei ist $\mu$ die Permeabilität und $\vec{H}$ das von außen angelegte Magnetfeld.
$\mu$ ergibt sich aus der Vakuum-Permeabilität $\mu_0$ und der materialabhängigen relativen Permeabilität $\mu_r$
\begin{equation}
    \mu = \mu_0 \cdot \mu_r .
\end{equation}

\subsection{Biot-Savart}
Jeder stromdurchflossener Leiter (z.B. Draht) ist von einem Magnetfeld umgeben. Dieses verläuft in Schleifen, die in einer Ebene
senktrecht zum Stromfluss verlaufen. Das Magnetfeld ergibt sich aus dem Biot-Savartschen Gesetz
\begin{equation}
    \symup{d}\vec{B} = \frac{\mu_0 I}{4\pi} \frac{\symup{d}\vec{s} \times \vec{r}}{r^3}
    \label{eqn:biot}
\end{equation}
mit der Magnetfeldstärke $\vec{B}$ bei dem Abstand $r$ vom Draht, welcher mit einem Strom $I$ durchflossen wird.

\subsection{Leiterschleifen}
Das Biot-Savart-Gesetz wird nun verwendet um das Magnetfeld einer Leiterschleife zu berechnen. Eine Leiterschleife
kann als Spule mit $n=1$ Windungen angesehen werden. Im Mittelpunkt des stromdurchflossenen Ringes folgt aus dem Biot-Savart-Gesetz\eqref{eqn:biot}
\begin{equation}
    \vec{B}(x) = \frac{\mu_0 I}{2} \frac{R^2}{(R^2+x^2)^(3/2)} \cdot \hat{x}
\end{equation}
für die Magnetfeldstärke. Hier beschreibt $x$ den Abstand zur Ringmitte und $R$ den Radius des Kreisrings.  \\

\subsection{Spulen}
Bei einem Solenoid (zylindrische Spule) verlaufen die Feldlinien innerhalb der Spule parallel zum Spulenrand. Das Magnetfeld ist in der Spulenmitte homogen und konstant.
Außerhalb der Spule verlaufen die Feldlinien in geschlossenen Schleifen. Das Feld ist hier inhomogen.\\
Die magnetische Feldstärke einer langen Spule berechnet sich durch
\begin{equation}
    B = \mu_r \mu_0 \frac{n}{l} I
\end{equation}
mit der Windungszahl $n$ und der Spulenlänge $l$.\\

\subsection{Helmholtz-Spule}
Wird ein homogenes Magnetfeld benötigt, so werden häufig zwei gleiche Spulen mit Radius $R$ verwendet.
Die Spulen stehen gegenüber mit dem Abstand $d=R$. Das Magnetfeld in der Mitte der Symmetrieachse eines Helmholtz-Spulenpaar mit $n=1$ Windungen wird durch
\begin{equation}
    B(0) = B_1(x) + B_1(-x) = \frac{\mu_0 I R^2}{(R^2+x^2)^{3/2}}
\end{equation}
beschrieben. Der allgemeine Fall für eine Spule mit $n$ Windungen und beliebigen Abstand $d$ wird durch die Formel %DAS FUNKTIONIERT NICHT: \cite[87--88]{demtroeder}
\begin{equation}
    B(z) = \frac{\mu_0 \cdot I \cdot R^2}{2} \cdot \left ( \frac{1}{\left((z+d/2)^2+R^2\right)^{3/2}} + \frac{1}{\left((z-d/2)^2+R^2\right)^{3/2}} \right )
    \label{eqn:helm}
\end{equation}
beschrieben.