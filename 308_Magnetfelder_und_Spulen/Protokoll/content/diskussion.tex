\section{Diskussion}
\label{sec:Diskussion}

Bei der Auswertung der langen Spule fällt auf, dass das vermutete Plateau nicht gemessen wird, da aufgrund der kurzen Hallsonde die Mitte der Spule nicht erreicht werden kann.
Allerings ist an der Grafik \ref{fig:long} zu erkennen, dass zwischen den gemssen Werten wahrscheinlich ein Plateau ist.

Bei der kurzen Spule konnte die Mitte der Spule erreicht werden. 
Allerdings ist kein Plateau zu messen, da durch die Länge der Spule die Randeffekte nicht vernachlässigt werden können.
Dabei ist die prozentuale Abweichung von experimentellem Wert und theoretischem Wert bei der langem Spule $7.188\%$ und bei der kurzen Spule $19.597\%$.
Die große Abweichung bei der kurzen Spule ist damit zu Begründen, dass bei der kurzen Spule die Randeffekte wesentlich größer sind als bei der langen Spule.

Bei dem Versuch mit den Helmholtzspulenpaar ist gut zu sehen, dass die Spulen nur bei einem bestimmten Abstand ein homogenes Feld aufbaut.
Dabei kommt der Abstand $d_2=7\SI{7}{\centi\meter}$ am nächsten an den optimalen Abstand heran, da wie in der Grafik \ref{fig:helm2} zu sehen ist, für diesen Abstand das Feld am homogensten ist.
Hier sind die Abweichung von Theoriewert und gemessenen Wert für $d_1$ $18.656\%$, für $d_2$ $15.256 \%$ und für $d_3$ $38.405 \%$.
Die Abweichung kommt dadruch zustande, dass der optimale Fall von $d=R$ nicht erreicht wird.

Da wir den Wert der Koerzitivkraft nicht direkt messen konnten, wurde dieser angenähert indem eine Verbindungslinie zwischen den zwei Werten gezogen wurde, die der Koerzitivkraft am nächsten sind.
Es fällt auf, dass die Hysteresekurve aufgrund des Materials des Kerns und ihrer Vorgeschichte schmaler als die typische Theoriekurve \ref{fig:hysterese} ist.