\section{Diskussion}
\label{sec:Diskussion}

Bei der Auswertung der langen Spule fällt auf, dass das vermutete Plateau nicht gemessen wird, da aufgrund der kurzen Hallsonde die Mitte der Spule nicht erreicht werden kann.
In Abbildung \ref{fig:long} sollte zwischen den gemessen Werten ein Plateau liegen.

Bei der kurzen Spule kann die Mitte der Spule erreicht werden. 
Allerdings ist kein Plateau zu messen, da durch die Länge der Spule die Randeffekte nicht vernachlässigt werden können.
Dabei ist die prozentuale Abweichung, des experimentellen Wert und theoretischen Wert der langen Spule und der kurzen Spule in der Tabelle \ref{tab:disk1} aufgetragen.
Die große Abweichung bei der kurzen Spule ist damit zu Begründen, dass bei der kurzen Spule die Randeffekte wesentlich größer sind als bei der langen Spule.

Bei dem Versuch mit den Helmholtzspulenpaar ist gut zu sehen, dass die Spulen nur bei einem bestimmten Abstand ein homogenes Feld aufbauen.
Dabei kommt der Abstand $d_2=\SI{7}{\centi\meter}$ am nächsten an den optimalen Abstand $d=R$ heran. Wie in der Grafik \ref{fig:helm2} zu sehen ist für diesen Abstand das Feld am homogensten.
Die relativen Abweichungen sowie Theoriewerte und gemessenene Werte sind für $d_1$, $d_2$ und $d_3$ in der Tabelle \ref{tab:disk2} zusammengefasst.
Die Abweichung kommt zustande, da der optimale Fall von $d=R$ nicht erreicht wird.

Da der Wert der Koerzitivkraft nicht direkt gemessen werden kann, wird dieser angenähert indem eine Verbindungslinie zwischen den zwei Werten gezogen wird, die der Koerzitivkraft am nächsten sind.
Es fällt auf, dass die Hysteresekurve aufgrund des Materials des Kerns und ihrer Vorgeschichte schmaler als die typische Theoriekurve \ref{fig:hysterese} verläuft.

\begin{table}
\centering
\caption{Messwerte, Theoriewert und relative Abweichung der kurzen und langen Spule.}
\begin{tabular}{c|ccc}
    \toprule
    Spulenart &  Messwert $B\:/\: \si{\milli\tesla}$ & Theoriewert $B\:/\: \si{\milli\tesla}$ & relative Abweichung in \% \\
    \midrule
    lange Spule & 1.837 & 1.993 & 7.188\\
    kurze Spule & 1.837 & 1.477 & 19.597 \\
    \bottomrule
\end{tabular}
\label{tab:disk1}
\end{table}


\begin{table}
\centering
\caption{Messwerte aufgetragen mit Theoriewerte und der relativen Abweichung bei der Helmholtzspule.}
    \begin{tabular}{c|ccc}
    \toprule
 Spulenabstand $d\:/\:\si{cm}$ &  Messwert $B\:/\: \si{\milli\tesla}$ & Theoriewert $B\:/\: \si{\milli\tesla}$ & relative Abweichung in \% \\
    \midrule
    $d_1= 11$ & 2.755 & 2.241 & 18.656\\
    $d_2= 7$ & 3.842 & 3.255 & 15.256 \\
    $d_3= 15$ & 1.919 & 2.656 & 38.405 \\
    \bottomrule
    \end{tabular}
    \label{tab:disk2}
\end{table}