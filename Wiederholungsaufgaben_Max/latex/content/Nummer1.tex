\section{Nummer 1}
\label{sec:nummer1}

\subsection{Mittelwert}
Der Mittelwert ist die Summe aus allen aufgenommenen Daten, geteilt durch die Anzahl dieser.
Dieser wird oft benutzt um den Einfluss durch experimentelle Fehler möglichst klein zu halten.

\begin{equation*}
    \bar x = \frac{\sum_\text{i}^N x_\text{i}}{N} 
\end{equation*}
Hierbei entspricht $x_\text{i}$ den einzelenen aufgenommen Werten, $\bar x$ dem Mittelwert und $N$ der Anzahl an aufgenommenen Werten.

Diese Art der Berechnung wird arithmetischer Mittelwert genannt,
 es gibt auch andere Arten der Berechnung, 
 diese werden für unsere Zwecke allerdings nicht so häufig genutzt.

\subsection{Standardabweichung}

Die Standardabweichung gibt an wie sehr die Werte eines Datensatzes von dem Mittelwert des Datensatzes abweichen.
Das Bedeutet, dass bei einer kleinen Standardabweichung sich die Werte eher in der Nähe des Mittelwertes befinden.
Wohingegen sie weiter weg vom Mittelwert sind wenn die Standardabweichung groß ist.
Die Standardabweichung wird berechnet durch

\begin{equation*}
    s = \sqrt{\frac{1}{N-1} \sum _{i=1}^N(x_i -\bar x)^2}.
\end{equation*}

$s$ ist hierbei die Standardabweichung, $N$ die Anzahl an Daten, $x_i$ sin die Daten und $\bar x$ ist der Mittelwert der Daten.

\subsection{Varianz und Standardabweichung}

Der Fehler des Mittelwertes entspricht der Standardabweichung.
Die Streuung der Messwerte, welche auch Varianz genannt wird, ist das Quadrat der Standardabweichung.
Die Varianz gewichtet die Werte, welche weiter vom Mittelwert weg liegen stärker, wodurch die Differenz zwischen Mittelwert und Wert stärker ins Gewicht fällt.
Die Standardabweichung gewichtet hingegen die Anzahl von Werten die vom Mittelwert abweichen stärker.


