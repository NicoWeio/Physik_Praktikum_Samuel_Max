\section{Theorie}
\label{sec:Theorie}

Zum Messen des Röntgenspektrums muss zunächst Strahlung erzeugt werden.
Dies geschieht in einer evakuierten Röhre, an dessen Enden sich eine Anode beziehungsweise eine Kathode befindet.
Diese stehen unter Spannung wodurch sich Elektronen aus der Kathode lösen und zur Anode hin beschleunigt werden.
Dort treffen sie auf und werden im Coulombfeld der Atome des Anodenmaterials gebremst.
Dadurch strahlen die Elektronen Photone ab welche sich als Bremsspektrum messen lassen.
Die Photonen haben dabei genau so viel Energie, wie die Elektronen beim abremsen verloren haben.
Wenn nun aber ein Elektron so auf ein Atom trifft, dass dieses ionisiert wird, also eine Leerstelle auf der Schale des Atoms entsteht.
Kann ein anderes Elektron von einer höheren Schale diese Leerstelle füllen.
Dabei gibt dieses Elektron Energie in Form von Photonen ab welche als charakteristisches Spektrum aufgenommen werden können.
Die Energie, die das Röntgenquant dabei hat, entspricht der Energiedifferenz 
\begin{equation}
    h \nu = E_ \text{m} - E_ \text{n} = \frac{ch}{\lambda},
    \label{eq:eng}
\end{equation}
also der Energie auf der höheren Schale $E_ \text{m}$ gegen die der tieferen Schale $E_\text{n}$.
Die verschiedenen Schalen auf denen sich die Elektronen bewegen werden mit K,L,M,... bezeichnet.
Zudem wird im Index angezeigt wo das Elektron her stammt, welches für die Austrahlung des Photons verantwortlich ist.
Im folgenden wird die Energie $E_{ \text{K},\alpha}$ der $K_\alpha$-Linie betrachtet werden.
Diese kann durch 
\begin{equation}
    E_{\text{K},\alpha} = R_ \infty (z - \sigma_1)^2 \frac{1}{1^2} - R_ \infty (z-\sigma_2)^2 \frac{1}{2^2}
    \label{eq:enkkante}
\end{equation}
berechnet werden.
Die Energie $E_{ \text{K},\beta}$ der $K_\beta$-Linie kann entsprechend der Gleichung
\begin{equation}
    E_{\text{K}, \beta} = R_ \infty (z - \sigma_1)^2 \frac{1}{1^2} - R_ \infty (z-\sigma_3)^2 \frac{1}{3^2}
\end{equation}
berechned werden.
Dazu müssen allerdings die Abschrimkonstanten $\sigma_1 $ und $\sigma_2$ sowie die Rydbergenergie $R_\infty$ und die Ordnungszahl $z$ des Materials bekannt sein.
Die Abschrirmkonstante an der $K$-Kante lässt sich durch 
\begin{equation}
    \sigma_ \text{K} = z - \sqrt{\frac{E_\text{K}}{R_\infty} - \frac{\alpha^2 z^4}{4}}
    \label{eq:sigma}
\end{equation}
berechnen.
Dabei entspricht $\alpha$ der Feinstrukturkonstante.
Zur Bestimmung der Rydbergfrequenz $R_\infty$ wird das Moseley'sche Gesetz
\begin{equation}
    E_\text{K} = R h (z - \sigma )^2
    \label{eqn:moseley}
\end{equation}
verwendet
\\\\
Zur Bestimmung der Wellenlänge des Spektrums wird ein LiF-Kristall nutzt, welcher die Strahlung in Richtung eines Geiger-Müller-Zählrohr reflektiert.
Durch die Struktur des Kristallgitters, interferiert die Strahlung je nach Winkel des Kristalls, konstruktiv oder destruktiv.
Bei einem bestimmten Winkel $\theta$, der Glanzwinkel genannt wird, kommt die konstruktive Interferenz zum Vorschein.
So kann durch
\begin{equation}
    2d\sin(\theta) = n \lambda
    \label{eq:bragg}
\end{equation}
die Wellenlänge der Strahlung bestimmt werden.



