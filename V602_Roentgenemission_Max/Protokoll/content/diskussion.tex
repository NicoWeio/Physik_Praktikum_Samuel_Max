\section{Diskussion}
\label{sec:Diskussion}
\subsection{Bragg-Bedinung}
Die Überprüfung der Bragg-Bedinung ergibt, dass diese $0.2\si{\degree}$ von der theoretischen Lage abweicht.
Die experimentell Bestimmte Lage befindet sich bei $\theta_\text{exp} = 28.2\si{\degree}$ wohingegen die theoretische bei $\theta_\text{theo}= 28\si{\degree}$ liegt.
Dies entspricht einer Abweichung von $0.7\%$.

\subsection{Emissionsspektrum der Cu-Röhre}
Zur Ermittlung der maximalen Energie des Bremsberges hätte eine größere Winkelspanne abgedeckt werden müssen, daher konnte diese nicht bestimmt werden.

\subsection{Absorptionsspektren}
In der Tabelle \ref{tab:absorption} sind die berechneten Absorptionsenergien und Abschrimkonstanten sowie die recherchierten Literaturwerte aufgelistet.
Die Abweichung von experimentellem Wert zum Literaturwerten sind in Tabelle \ref{tab:abweichung} zu finden.
Die berechnete Rydbergenergie $R_{\infty,\text{exp}}= \SI{12.47(20)}{\eV}$ weicht von dem Wert welcher in der Literatur gegeben ist $R_\infty = \SI{13.6}{\eV}$ um $8.3\%$ ab.
Die Abweichung sind den experimentellen Umständen entsprechend.
Sie könnten noch geringer ausfallen, wäre der Aufbau besser vor äußerer Strahlung geschützt und wären nicht so viele Näherungen gemacht worden.
Besonders durch die Verbesserung der Bragg-Bedingung hätte ein besseres Ergebnis erziehlt werden können.

\begin{table}
    \centering
    \caption{Abweichungen der berechneten Werte zum Literaturwert}
    \begin{tabular}{c c c}
        \toprule
        Element & $\Delta E\,/\,\si{\percent}$ & $ \Delta\sigma \,/\, \% $ \\
        \midrule
        Zn & 0.51 & 1.68 \\
        Ga & 0.74 & 2.77 \\
        Br & 0.06 & 0.51 \\
        Rb & 0.97 & 4.06 \\
        Sr & 0.69 & 3.00 \\
        Zr & 1.51 & 6.84 \\
        \bottomrule
    \end{tabular}
    \label{tab:abweichung}
\end{table}


\begin{table}
    \centering
    \caption{Ergebnisse und Literaturwerte der Absorptionsenergie und der Abschirmkonstante.\cite{xray}}
    \sisetup{
        table-format=2.2
    }
    \begin{tabular}{c S[table-format=1.0] S S S S}
        \toprule
        Element & $Z$ & $\frac{E}{10^{-18}\si{\J}}$ & $\frac{E_\text{Lit}}{10^{-18}\si{\J}}$ & $\sigma$ & $\sigma_\text{Lit}$ \\
        \midrule
        Zn & 30 & 1.5381 & 1.5461 & 3.63 & 3.57 \\
        Ga & 31 & 1.6491 & 1.6614 & 3.71 & 3.61 \\
        Br & 35 & 2.1596 & 2.1581 & 3.83 & 3.85 \\
        Rb & 37 & 2.4115 & 2.4353 & 4.10 & 3.94 \\
        Sr & 38 & 2.5615 & 2.5795 & 4.12 & 4.00 \\
        Zr & 40 & 2.8399 & 2.8839 & 4.37 & 4.09 \\
        \bottomrule
    \end{tabular}
    \label{tab:absorption}
\end{table}

