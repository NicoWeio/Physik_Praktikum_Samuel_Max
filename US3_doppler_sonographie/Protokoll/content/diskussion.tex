\section{Diskussion}
\label{sec:Diskussion}

Da Aufgrund der Fehlfunktion der Pumpe die konnte leider die Fließgeschindigkeit der Flüssigkeit nicht bestimmten werden.
Dies hat den Rest der Auswertung aber kaum beeinflusst.
Der Verlauf der Frequenzdifferenz in Abhängigkeit von $N$, zeigt, dass bei höherer Umdrehungsgeschwindkeit die Frequenzdifferenz abnimmt.
Das ist damit zu erklären, dass sich die Flüssigkeit in dem Rohr in die Richtung bewegt, in die sich auch der Schall der Sonde ausbreitet.
Demnach folgen die Messwerte den Erwartungen und der Doppler-Effekt führt so bei höherer Geschwindigkeit zu einer geringeren Frequenzverschiebung.
\\\\
In der Abbildungen \ref{fig:3880} vom zweiten Teil der Auswertung ist, dass die Messwerte der Streuintensität sehr ungenau sind.
Diese hätten theoretisch konstant bleiben müssen.
Allerdings war es sehr schwer die Streuintensität zu messen, da die Angaben an dem Messgerät schon bei kleinsten Störungen große Unterschiede aufgewiesen haben.
Wenn zwischen der Sonde zum Beispiel und dem Prisma  nur ein paar Luftblasen waren, konnten schon kleinste Bewegungen der Sonde die Streuintensität stark verändern.
Das selbe gilt für die Messung der Streuintensität für $N=\SI{6000}{rpm}$.
\\\\
Bei der Messung der maxmimalen Frequenz fällt auf, dass diese bis zu einer bestimmten Messtiefe steigt und ab dort wieder abfällt.
Bei der Messung mit $N=\SI{3880}{rpm}$, dessen Messwerte in Abbildung \ref{fig:3880} zu sehen sind, wird das Maximum bei einer Messtiefe von $\SI{6}{\milli\meter}$ erreicht.
Die maximale Frequenz für $N=\SI{6000}{rpm}$ wird bei ungefähr $\SI{7}{\milli\meter}$ erreicht.
Die Maxima der maximalen Frequenz sind also für beide Umdrehungsgeschwindigkeiten recht nah beieinander.
Aus dem restlichen Verlauf der beiden Messungen ist zu schließen, dass die Geschwindigkeit der Flüssigkeit und somit auch die maximale Frequenz bis zu einer Messtiefe von $6-7 \,\si{\milli\meter}$ zunimmt.
Nach diesem Punkt nimmt sie wieder ab.
Also ist die Geschwindigkeit der Flüssigkeit an der Rändern der Röhre niedriger als in der Mitte.
Da bei beiden für beide Umdrehungsgeschwindigeiten das selbe Rohr genutzt wurde ist es nicht verwunderlich, dass beide Frequenzverläufe starke Ähnlichkeiten aufweisen.
Auch für die Messung der Frequenz waren allerdings schon kleine Änderung der Position der Sonde für recht große Unterschiede in den Werten verantwortlich.
Damit sind auch die Sprünge in den Verläufen zu erklären die bei beiden Werte für $N$ auftreten.