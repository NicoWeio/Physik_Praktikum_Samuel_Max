\section{Diskussion}
\label{sec:Diskussion}
Die Untergrundrate beträgt $N_U = \SI{139(4)}{\frac{Imp}{\second}}$.
Um eine zuverlässige Untergrundrate zu erhalten, muss diese häufiger oder in einem längeren Zeitintervall gemessen werden.
\\
Die gemessenen Werte und die Literaturwerte werden in Tab. \ref{tab:ergebnisse} verglichen.
Der Übergang zwischen schnellem und langsamen Zerfall ist nicht eindeutig, daher sind geringe Abweichungen bei Rhodium zu erwarten.
Die Abweichung zu den Theoriewerten \cite{zerfall} ist relativ gering und die Ergebnisse entsprechen der Literatur.
\begin{table}
    \centering
    \begin{tabular}{c|cccc}
        \toprule
        Zerfall & $T_\text{exp} \,/\, \si{\second}$ & $T_\text{th} \,/\, \si{\second}$ & Abs. Abweichung $\,/\, \si{\second}$ & Rel. Abweichung $\,/\, \si{\percent}$ \\
        \midrule
        $\ce{^{52}_{23}V}$ & $\SI{215(7)}{}$ & $224.58$ & $9.58$ & $4.2$ \\
        $\ce{^{104i}_{45}Rh}$ & $\SI{257(37)}{}$ & $260.4$ & $3.4$ & $1.3$ \\
        $\ce{^{104}_{45}Rh}$ & $\SI{43.0(11)}{}$ & $42.3$ & $0.7$ & $1.7$ \\
        \bottomrule
    \end{tabular}
    \caption{Experimentell gemessene und Literaturwerte \cite{zerfall} im Vergleich.}
    \label{tab:ergebnisse}
\end{table}
