\section{Diskussion}
\label{sec:Diskussion}
Die Kurven für einen Heizstrom von $I_f = \SI{2.0}{\ampere}$ und $I_f = \SI{2.2}{\ampere}$ haben den typischen Verlauf (siehe Abb. \ref{fig:kurvenchar}).
Die Kurven flachen relativ schnell ab und nähern sich dem Sättigungsstrom $I_S$ an.
Auffällig ist, dass die Kurve bei einem Heizstrom von $I_f=\SI{2.4}{\ampere}$ kaum abflacht und der Sättigungswert erst bei sehr hohen Spannungen erreicht wird.
\\
Ingesamt verlaufen die Kurven exponentiell, wie zu erwarten war.
\\\\
Das Lanmuir-Schottkysche Raumladungsgesetz besagt, dass $I \sim U^{\frac{3}{2}}$ gilt.
Der experimentell bestimmte Exponent
\begin{equation*}
    b = \num{1.521(11)}
\end{equation*}
weicht um $\SI{1.4}{\percent}$ vom Raumladungsgesetz ab.
Der Verlauf der Kurve entspricht der Theorie und auch die Messwerte liegen in der Ausgleichskurve (siehe Abb. \ref{fig:langmuir}).
\\\\
Die Kathodentemperatur für die maximale Heizleistung wurde zum einen über eine Ausgleichsrechnung im Anlaufstromgebiet $T_1$ bestimmt und zum anderen mithilfe einer Leistungsbilanz des Heizstromkreises $T_2$:
\begin{align*}
    T_1 &= \SI{1950(10)}{\kelvin} \\
    T_2 &= \SI{2161.80}{\kelvin}
\end{align*}
Die Abweichung beträgt $\SI{211.8}{\kelvin}$.
\\\\
Die experimentell bestimmte Austrittsarbeit für Elektronen in Wolfram beträgt
\begin{equation*}
    \phi_\text{exp} = \SI{3.7327(27)}{\electronvolt}
\end{equation*}
und weicht somit um $\SI{17.05}{\percent}$ vom Theoriewert \cite{austrittsarbeit}
\begin{equation*}
    \phi_\text{th} = \SI{4.5}{\electronvolt}
\end{equation*}
ab.
Das Experiment ist sehr empfindlich, da z.B im nano-Bereich gemessen wird und somit kleinste Störungen die Ergebnisse beeinflussen.
Schon während des Experiments fiel auf, dass durch einen zu geringen Abstand zur Kathode der gemessene Strom stark schwankt.
Um bessere Ergebnisse zu erhalten, sollte das Experiment vor äußeren Einflüssen abgeschirmt werden.
