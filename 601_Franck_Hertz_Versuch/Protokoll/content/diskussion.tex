\section{Diskussion}
\label{sec:Diskussion}
Das Verhältnis $\frac{a}{\bar{w}}$ der mittleren freien Weglänge $\bar{w}$ und dem Abstand $a$ zwischen Kathode und Beschleunigungselektrode sollte zwischen $1000 - 4000$ liegen, damit eine ausreichende Stoßwahrscheinlichkeit gegeben ist.
Dies ist nur für die Temperatur $T = \SI{140}{\degreeCelsius}$ gegeben.
Bei der Messung bei Raumtemperatur ist davon auszugehen, dass die Elektronen ohne Wechselwirkung von der Kathode zur Auffängerelektrode durchlaufen sind, da das Verhätnis weit unter $1000$ liegt.
Das Verhältnis ist für die Temperatur $T = \SI{180}{\degreeCelsius}$ etwas zu groß und es besteht die Möglichkeit, dass nur wenige Elektronen die Auffängerelektrode überhaupt erreichen.
\\
\\
Die integrale Energieverteilung entspricht nicht der Erwartung.
Für beide Temperaturen verlaufen die Kurven relativ linear und flachen kaum ab.
Die differentielle Energieverteilung ist daher relativ konstant mit großer Streuung.
\\
Durch die geringe Anzahl an Messwerten, da kein XY-Schreiber vorhanden, ist kein kontinuierlicher Verlauf der Energieverteilung möglich.
Eine Messung mit mehr Messwerten, also kleinerem $\Delta U_A$ oder einem XY-Schreiber würde genauere Ergebnisse bringen.
\\
\\
Die Franck-Hertz-Kurve \ref{fig:franck_hertz} entspricht der Erwartung.
Die Maxima steigen mit wachsendem $U_A$ und auch die Abstände zwischen den Maxima
\begin{equation*}
    \Delta \bar{U}_{B,max} = \SI{5.29(18)}{\volt}
\end{equation*}
sind mit einer Abweichung von $\SI{3.4}{\percent}$ konstant.
\\
Die experimentell ermittelte Anregungsenergie
\begin{equation*}
    U_{1,exp} = \SI{5.29(18)}{\electronvolt}
\end{equation*}
weicht um $\SI{7.96}{\percent}$ vom Literaturwert \cite{anregungsenergie} ${U_{1,th} = \SI{4.9}{\electronvolt}}$ ab.