\section{Durchführung}
\label{sec:Durchführung}

Zur Messung der Spannung wird ein digitales Oszilloskop benutzt.

\subsection{Wheatstonesche Brücke}

Zunächst wird die Wheatstonesche Brücke wie in Abbildung \ref{fig:wheatstonesche} dargestellt aufgebaut.
Danach wird ein bekannter Widerstand an der Stelle des Widerstands $R_2$ eingesetzt.
Nun wird das Widerstandsverhältnis des Potentiometers, also $R_3$ und $R_4$, solange variiert, bis mit dem Oszilloskop zwischen Punkt $A$ und $B$ keine Brückenspannung mehr gemessen wird.
Daraufhin werden die Widerstände für $R_3$ und $R_4$ notiert und $R_2$ durch einen neuen bekannten Widerstand ausgetauscht.
Dieses Prozedere wird drei mal durchgeführt.
Danach wird ein neuer unbekannter Widerstand $R_x$ eingebaut und der Versuch wird wie oben beschrieben erneut durchgeführt.

\subsection{Kapazitätsmessbrücke}

Die Kapazitätsmessbrücke wird nach Abbildung \ref{fig:kapaz} aufgebaut.
In dem Aufbau ist der Widerstand $R_2$ ein fester bekannter Widerstand.
Nun wird wieder das Widerstandsverhältnis zwischen $R_3$ und $R_4$ variiert, bis keine Brückspannung mehr messbar ist.
Daraufhin werden die Widerstandswerte von $R_3$ und $R_4$ notiert und der Kondensator $C_2$ ausgetauscht.
Dieser Prozess wird zwei mal wiederholt.

\subsection{Induktivitätsmessbrücke}

Nach Abbildung \ref{fig:induk} wird nun die Induktivitätsmessbrücke aufgebaut.
Auch in diesem Aufbau wird anstatt eines variablen Widerstand $R_2$ ein fester Widerstand verwendet.
Nun ist das Prozedere wie bei den beiden Brücken zuvor. Die Widerstände $R_3$ und $R_4$ werden verändert, solang bis die Brückspannung nicht mehr messbar ist.
Dann wird das Verhältnis notiert und der Widerstand $R_2$, der sich hinter der Spule $L_2$ befindet, ausgetauscht.
Es werden insgesamt drei verschiedene Widerstände eingesetzt und die jeweiligen Werte notiert.

\subsection{Maxwell-Brücke}

Nun werden wieder Induktivitäten gemessen, diesemal aber nach Aufbau der Maxwell-Brücke die nach Abbildung \ref{fig:maxwell} aufgebaut wird.
Nach dem Aufbau wird das Widerstandsverhältnis von $R_3$ und $R_4$ variiert bis ein Minimum der Brückenspannung gefunden wird, dann das Verhältnis notiert und der Kondensator $C_4$ ausgetauscht.
Daraufhin wird der Prozess mit zwei weiteren Kondensatoren wiederholt.

\subsection{Wien-Robinson-Brücke}

Die Wien-Robinson-Brücke wird nun nach Abbildung \ref{fig:wien} aufgebaut. 
Nun wird die Speisespannung $U_\text{S}$, mit der die Wien-Robinson-Brücke gespeist wird, gemessen.
Danach wird das Oszilloskop wieder an die Wien-Robinson-Brücke angeschlossen um die Brückenspannung zu messen.
Nun wird am Generator die Frequenz des Wechselstroms langsam verändert, wobei die Änderung in der Umgebung des Minimums der Brückspannung am geringsten gewählt wird um eine möglichst hohe Genauigkeit zu erzielen. 
Es werden für die spätere Auswertung 38 Wertepaare aufgenommen und notiert.



 
