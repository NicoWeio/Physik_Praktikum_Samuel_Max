\section{Durchführung}
\label{sec:Durchführung}

\subsection{Dewar-Gefäß}
Zunächst muss die Wärmekapazität des Dewar-Gefäß bestimmt werden.
Dafür werden $\SI{300}{\gram}$ Wasser in das Gefäß gegeben und die Temperatur des Wassers gemessen.
Daraufhin werden weitere $\SI{300}{\gram}$ Wasser auf $\Si{80}{\celsius}$ erhitzt und ebenfalls in das Gefäß gegeben.
Nun wird das Dewar-Gefäß so gut es geht verschlossen, während die Temperatur des Wasser im Gefäß mit einem Thermometer gemessen wird.
Nach ungefähr 2 Minuten wir die Temperatur des Wasser, welches inzwischen eine homogene Temperatur haben sollte, gemessen und notiert.
Mit der Differenz zwischen Start Temperatur des Wassers und Mischtemperatur des Wasser nach zwei Minuten, kann nun berechnet werden wie viel Wärme das Dewar-Gefäß aufgenommen hat.

\subsection{Messungen der Proben}
Nun werden $\SI{600}{\gram}$ Wasser in das Dewar-Gefäß gegeben.
Dieses sollte eine Temperatur von $\SI{22}{\celsius}$ nicht überschreiten, damit später ein gut messbarer Temperaturunterschied vorhanden ist.
Die Temperatur des Wasser wird mit einem Thermometer bestimmt und notiert.
Ein weiteres Gefäß wird mit Wasser gefüllt. 
In dieses wird eine Metall Probe gegeben.
Das Gefäß wird dann auf eine Heizplatte gestellt, sodass sich das Wasser und die darin enthaltene Probe erhitzt.
Bei einer Temperatur von $\SI{90}{\celsius}$ wird die Probe aus dem Behälter genommen und in das Dewar-Gefäß gelegt.
Das Gefäß wird mit einem Deckel ver