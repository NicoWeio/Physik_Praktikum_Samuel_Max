\section{Diskussion}
\label{sec:Diskussion}
Die Theoriewerte für $c_\text{k}$ befinden sich in Tab. \ref{tab:vergleich}. \cite{konstanten} \\ 
Die Molwärme beträgt nach der Theorie $3R \approx \SI{24.94}{\frac{\joule}{\mol\kelvin}}$.
Zudem sind in Tab. \ref{tab:vergleich} und \ref{tab:vergleich_cv} die experimentell gemessenenen Werte und deren Abweichungen zu finden.

\begin{table}
    \centering
    \begin{tabular}{c|S[table-format=1.4+-1.3] S[table-format=1.3] S[table-format=2.1]}
        \toprule
        Material & $c_\text{k,exp}\, / \,\si{\frac{\joule}{\gram\kelvin}}$ & $c_\text{k,th}\, /\, \si{\frac{\joule}{\gram\kelvin}}$ & \text{Abw.} $c_\text{k}\, /\, \% $ \\
        \midrule
        Graphit & $\SI{0.80(6)}{}$ & 0.710   & 12.7  \\
        Blei & $\SI{0.1233(31)}{}$ & 0.131  & 5.9 \\
        Aluminium & $\SI{0.840}{}$ & 0.896 & 6.3 \\
        \bottomrule
    \end{tabular}
    \caption{Theorie- und Praxiswerte der Wärmekapazität $c_\text{k}$ und Molwärme $C_\text{V}$ im Vergleich.}
    \label{tab:vergleich}
\end{table}

\begin{table}
    \centering
    \begin{tabular}{c | S[table-format=1.4+-1.3] S[table-format=2.1]}
        \toprule
        Material & $C_\text{V,exp} \,/\, \si{\frac{\joule}{\mol\kelvin}}$ & \text{Abw.} $C_\text{V,exp} \,/\, \si{\frac{\joule}{\mol\kelvin}}$ \\
        \midrule
        Graphit & $\SI{9.6(7)}{}$ & 61.6 \\
        Blei & $\SI{23.8(7)}{}$ & 4.5 \\
        Aluminium & $\SI{10.04}{}$ & 59.7 \\
        \bottomrule
    \end{tabular}
\caption{Theorie- und Praxiswerte der Molwärme $C_\text{V}$ im Vergleich.}
\label{tab:vergleich_cv}
\end{table}

Auffällig ist, dass die Werte für die Molwärme besonders bei den Stoffen stark abweicht, die eine geringe Dichte haben.
Bei dem Versuch waren diese Stoffe Aluminium und Graphit.
Deswegen ist zu vermuten, dass bei diesen Stoffen die Temperatur, welche zur Übereinstimmung des klassichem Modell mit der Quantenmechanik nötig ist, nicht erreicht wurde.
Blei, welche eine hohe Dichte hat, konnte diese Temperatur erreicht werden. 
Der gemessene Wert für die Molwärme stimmt in diesem Fall mit der Theorie überein.
Es ist auch nicht davon auszugehen, dass die Molwärme der beiden Stoffe Graphit und Aluminium experimentell falsch bestimmt wurde.
Denn die spezifische Wärmekapazität $c_\text{k}$ dieser beiden Stoffe, aus der später die Molwärme berechnet wird, weicht nur geringfügig von den Literaturwerten ab.
Also lässt sich abschließend sagen, dass das klassische Modell nur in spezifischen Fällen die Praxis passent beschreibt.
