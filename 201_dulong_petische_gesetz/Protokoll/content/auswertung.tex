\section{Auswertung}
\label{sec:Auswertung}
\subsection{Wärmekapazität des Kalorimeters}
Das Wasser mit Masse $m_\text{x}$ und Temperatur $T_\text{x}$ befindet sich im Kalorimetergefäß.
Ein weiteres Gefäß mit Wasser mit der Masse $m_\text{y}$ wird auf die Temperatur $T_\text{y}$ erwärmt.
Der Literaturwert für die Wärmekapazität von Wasser beträgt $c_\text{w} = \SI{4.18}{\joule / \gram\kelvin}$ \cite[159]{anleitung}.
Es wurden folgende Werte gemessen:
\begin{equation*}
    m_\text{x} = \SI{263.94}{\gram}
\end{equation*}
\begin{equation*}
    m_\text{y} = \SI{286.95}{\gram}
\end{equation*}
\begin{equation*}
    T_\text{x} = \SI{294.75}{\kelvin}    
\end{equation*}
\begin{equation*}
    T_\text{y} = \SI{353.15}{\kelvin}
\end{equation*}
\begin{equation*}
    T_\text{m} = \SI{323.15}{\kelvin}
\end{equation*}


Nach \eqref{eqn:waerme_vol} folgt für die Wärmekapazität
\begin{equation}
    c_\text{g}m_\text{g} = \SI{163.76}{\joule/\kelvin} .
\end{equation}

\subsection{Spezifische Wärmekapazität und Atomwärme}
Der Probekörper mit Masse $m_\text{k}$ wird auf die Temperatur $T_\text{k}$ erwärmt.
Dann wird der Körper in das Kalorimetergefäß mit der Masse $m_\text{Gefäß}$ getaucht.
Das Kalorimetergefäß ohne Wasser hat die Masse $m_\text{Glas}$.
Also hat das Wasser die Masse $m_\text{w} = m_\text{Gefäß}-m_\text{Glas}$ mit der Anfangstemperatur $T_\text{w}$.
Nach einer bestimmten Zeit stellt sich die Mischungstemperatur $T_\text{m}$ ein.
\subsubsection{Graphit}
Die 3 Messungen für die Probe Graphit ergeben die Werte aus Tab. \ref{tab:graphit}.
\begin{table}
    \centering
    \csvreader[tabular=c|cccccc,
    head=false,
    table head= Messung & $m_\text{k} / \si{\gram}$ & $m_\text{Gefäß} / \si{\gram}$ & $m_\text{Glas} / \si{\gram}$ & $T_\text{k} / \si{\celsius}$ & $T_\text{w} / \si{\celsius}$ & $T_\text{m} / \si{\celsius}$\\
    \midrule,
    late after line= \\]
    {content/data/graphit.csv}{1=\eins, 2=\zwei, 3=\drei, 4=\vier, 5=\fuenf, 6=\sechs, 7=\sieben}{$\num{\eins}$ & $\num{\zwei}$ & $\num{\drei}$ & $\num{\vier}$ & $\num{\fuenf}$ & $\num{\sechs}$ & $\num{\sieben}$}
    \caption{Die gemessenen Daten zur Probe Graphit. }
    \label{tab:graphit}  
\end{table}
Nach \eqref{eqn:dewar} folgt für die spezifische Wärmekapazität des Probekörpers:
\begin{equation}
    c_\text{k,1} = \SI{0.678}{\joule/\gram\kelvin}
\end{equation}
\begin{equation}
    c_\text{k,2} = \SI{0.850}{\joule/\gram\kelvin}
\end{equation}
\begin{equation}
    c_\text{k,3} = \SI{0.876}{\joule/\gram\kelvin}
\end{equation}
Um den Mittelwert zu ermitteln wird
\begin{equation}
    \mu = \frac{1}{n} \sum_{i=1}^n x_i
    \label{eqn:mittel}
\end{equation}
verwendet. Hier wird das Python-Plugin Numpy \cite{numpy} verwendet.
Wobei $x_i$ der $i$-te Wert bei $n$ Werten ist.
Um den Fehler zu berechnen wird
\begin{equation}
    \sigma = \sqrt{\frac{1}{n(n-1)} \sum_{i=1}^n (x_i - \mu)^2}
    \label{eqn:fehler}
\end{equation}
verwendet. Hier wird das Python-Plugin Numpy \cite{numpy} verwendet.
Werden fehlerbehaftete Größen in Formeln verwendet, so wird im Folgenden die Gauß'sche Fehlerfortpflanzung 
\begin{equation}
    \Delta y = \left|\frac{\partial y}{\partial x_1}\right| \Delta x_1 + \left|\frac{\partial y}{\partial x_2}\right| \Delta x_2 + ...
\end{equation}
verwendet. Die $\Delta$-Werte beschreiben die Fehlergrenzen.
Die Fehler werden im Folgenden mithilfe des Python-Plugin uncertainties \cite{uncertainties} berechnet.
\\
Im Mittel \eqref{eqn:mittel} mit dem Fehler \eqref{eqn:fehler} beträgt die spezifische Wärmekapazität für Blei
\begin{equation}
    c_\text{k,Graphit} = \SI{0.80(6)}{\joule/\gram\kelvin} .
\end{equation}
Alle folgenden Mittelwerte werden nach Gleichung \eqref{eqn:mittel} mit dem dazugehörigen Fehler \eqref{eqn:fehler} berechnet.
\\
Die Molwärme ergibt sich aus dem Zusammenhang zwischen $C_\text{P}$ und $C_\text{V}$ \eqref{eqn:waerme_vol}.
Die Werte für $\alpha$, $M$, $\kappa$ und $\rho$ werden aus der Anleitung entnommen. \cite[159]{anleitung}
Für den Probekörper aus Graphit ergeben sich folgende Atomwärmen:
\begin{equation*}
    C_\text{V,1} = \SI{8.10}{\frac{\joule}{\mol\kelvin}}
\end{equation*}
\begin{equation*}
    C_\text{V,2} = \SI{10.17}{\frac{\joule}{\mol\kelvin}}
\end{equation*}
\begin{equation*}
    C_\text{V,3} = \SI{10.48}{\frac{\joule}{\mol\kelvin}}
\end{equation*}
Im Mittel ergibt sich eine Molwärme von
\begin{equation}
    C_\text{V,Graphit} = \SI{9.6(7)}{\frac{\joule}{\mol\kelvin}} .
\end{equation}

\subsubsection{Blei}
Für den Probekörper aus Blei ergeben sich die Werte aus Tab. \ref{tab:blei}.
Daraus folgt nach \eqref{eqn:kapa} für die spezifische Wärmekapazität:
\begin{equation*}
    c_\text{k,1} = \SI{0.123}{\joule/\gram\kelvin}
\end{equation*}
\begin{equation*}
    c_\text{k,2} = \SI{0.129}{\joule/\gram\kelvin}
\end{equation*}
\begin{equation*}
    c_\text{k,3} = \SI{0.118}{\joule/\gram\kelvin}
\end{equation*}
Im Mittel ergibt sich der Wert
\begin{equation}
    c_\text{k,Blei} = \SI{0.1233(31)}{\joule/\gram\kelvin}.
\end{equation}
\begin{table}
    \centering
    \csvreader[tabular=c|cccccc,
    head=false,
    table head= Messung & $m_\text{k} / \si{\gram}$ & $m_\text{Gefäß} / \si{\gram}$ & $m_\text{Glas} / \si{\gram}$ & $T_\text{k} / \si{\kelvin}$ & $T_\text{w} / \si{\kelvin}$ & $T_\text{m} / \si{\kelvin}$\\
    \midrule,
    late after line= \\]
    {content/data/blei.csv}{1=\eins, 2=\zwei, 3=\drei, 4=\vier, 5=\fuenf, 6=\sechs, 7=\sieben}{$\num{\eins}$ & $\num{\zwei}$ & $\num{\drei}$ & $\num{\vier}$ & $\num{\fuenf}$ & $\num{\sechs}$ & $\num{\sieben}$}
    \caption{Die gemessenen Daten zur Probe Blei. }
    \label{tab:blei}  
\end{table}
\\
Die Atomwärme für die jeweiligen Messungen des Probekörpers aus Blei haben folgende Werte:
\begin{equation*}
    C_\text{V,1} = \SI{23.74}{\frac{\joule}{\mol\kelvin}}
\end{equation*}
\begin{equation*}
    C_\text{V,2} = \SI{25.00}{\frac{\joule}{\mol\kelvin}}
\end{equation*}
\begin{equation*}
    C_\text{V,3} = \SI{25.74}{\frac{\joule}{\mol\kelvin}}
\end{equation*}
Im Mittel ergibt sich eine Molwärme von
\begin{equation}
    C_\text{V,Blei} = \SI{23.8(7)}{\frac{\joule}{\mol\kelvin}} .
\end{equation}

\subsubsection{Aluminium}
Die Messung für Aluminium ergibt folgende Massen und Temperaturen:
\begin{equation*}
    m_\text{p} = \SI{151.85}{\gram}
\end{equation*}
\begin{equation*}
    m_\text{D} = \SI{584.12}{\gram}
\end{equation*}
\begin{equation*}
    T_\text{p} = \SI{363.15}{\kelvin}
\end{equation*}
\begin{equation*}
    T_\text{D} = \SI{294.55}{\kelvin}
\end{equation*}
\begin{equation*}
    T_\text{m} = \SI{297.75}{\kelvin}
\end{equation*}
Die spezifische Wärmekapazität für Aluminium beträgt
\begin{equation}
    c_\text{k,Alu} = \SI{0.840}{\joule/\gram\kelvin} .
\end{equation}
Für die Molwärme folgt nach \eqref{eqn:waerme_vol}
\begin{equation}
    C_\text{V,Alu} = \SI{21.56}{\frac{\joule}{\mol\kelvin}} .
\end{equation}