\section{Auswertung}
\label{sec:Auswertung}
\subsection{Wärmekapazität des Kalorimeters}
Das Wasser mit Masse $m_\text{x}$ und Temperatur $T_\text{x}$ befindet sich im Kalorimetergefäß.
Ein weiteres Gefäß mit Wasser mit der Masse $m_\text{y}$ wird auf die Temperatur $T_\text{y}$ erwärmt.
Es wurden folgende Werte gemessen:
\begin{equation*}
    m_\text{x} = \SI{263.94}{\gram}
\end{equation*}
\begin{equation*}
    m_\text{y} = \SI{286.95}{\gram}
\end{equation*}
\begin{equation*}
    T_\text{x} = \SI{294.75}{\kelvin}    
\end{equation*}
\begin{equation*}
    T_\text{y} = \SI{353.15}{\kelvin}
\end{equation*}
\begin{equation*}
    T_\text{m} = \SI{323.15}{\kelvin}
\end{equation*}

Nach Gleichung \eqref{eqn:ref} folgt für die Wärmekapazität
\begin{equation}
    c_\text{g}m_\text{g} = \SI{163.76}{\joule/\kelvin} .
\end{equation}

\subsection{Spezifische Wärmekapazität}
\subsubsection{Graphit}
\begin{table}
    \centering
    \csvreader[tabular=c|cccccc,
    head=false,
    table head= Messung & $m_\text{k} / \si{\kg}$ & $m_\text{Gefäß} / \si{\kg}$ & $m_\text{Glas} / \si{\kg}$ & $T_\text{k} / \si{\kelvin}$ & $T_\text{w} / \si{\kelvin}$ & $T_\text{m} / \si{\kelvin}$\\
    \midrule,
    late after line= \\]
    {content/data/graphit.csv}{1=\eins, 2=\zwei, 3=\drei, 4=\vier, 5=\fuenf, 6=\sechs, 7=\sieben}{$\num{\eins}$ & $\num{\zwei}$ & $\num{\drei}$ & $\num{\vier}$ & $\num{\fuenf}$ & $\num{\sechs}$ & $\num{\sieben}$}
    \caption{Die gemessenen Daten zur Probe Graphit. }
    \label{tab:graphit}  
\end{table}

\subsubsection{Blei}
\label{sec:Auswertung}
\begin{table}
    \centering
    \csvreader[tabular=c|cccccc,
    head=false,
    table head= Messung & $m_\text{k} / \si{\kg}$ & $m_\text{Gefäß} / \si{\kg}$ & $m_\text{Glas} / \si{\kg}$ & $T_\text{k} / \si{\kelvin}$ & $T_\text{w} / \si{\kelvin}$ & $T_\text{m} / \si{\kelvin}$\\
    \midrule,
    late after line= \\]
    {content/data/blei.csv}{1=\eins, 2=\zwei, 3=\drei, 4=\vier, 5=\fuenf, 6=\sechs, 7=\sieben}{$\num{\eins}$ & $\num{\zwei}$ & $\num{\drei}$ & $\num{\vier}$ & $\num{\fuenf}$ & $\num{\sechs}$ & $\num{\sieben}$}
    \caption{Die gemessenen Daten zur Probe Blei. }
    \label{tab:blei}  
\end{table}