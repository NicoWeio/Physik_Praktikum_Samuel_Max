\section{Diskussion}
\label{sec:Diskussion}

Es fällt zunächst auf, dass der berechnete $RC$ Wert der linearen Ausgleichrechnung \ref{sec:met1} aus dem Muster fällt.
Dieser Wert ist in der ersten Zeile der Tabelle \ref{tab:fehler} zu finden.
Der Wert weicht stark von den anderen Werten ab, die in der zweiten Spalte der Tabelle \ref{tab:fehler} zu finden sind.
Wobei die relativen Abweichungen aller Fehler recht gering sind.
Die Messwerte alle Versuchsteile zeigen Verläufe die erwartet waren.
Die Phsenverschiebung bleibt zwischen den erwarteten Werten von $0$ und $\frac{\pi}{2}$, und auch der frequenzabhängige Verlauf entspricht der Theorie.
Die Abweichungen die in der Auswertung auftreten sind mit Fehlern beim ablesen zu begründen, die leider bei analogen Osziloskopen nicht zu verhindern sind.


\begin{table}
\centering
\caption{Berechneten Werte im Vergleich zur Abweichung und relativem Fehler.}
\begin{tabular}{c|cc}
    \toprule
    Methode & Berechneter Wert $ \:/\: \si{\milli\second} $ & relativer Fehler \:/\: \% \\
    \midrule
    \ref{sec:met1} & $RC = \SI{0.5969(229)}{}$ & 3.8364 \\
    \ref{sec:met2} & $RC = \SI{1.2360(167)}{}$ & 1.3511 \\
    \ref{sec:met3} & $RC = \SI{1.2685(214)}{}$ & 1.6870 \\ 
    \bottomrule
\end{tabular}
\label{tab:fehler}
\end{table}
