\section{Diskussion}
\label{sec:Diskussion}

Es fällt zunächst auf, dass der berechnete $RC$ Wert der linearen Ausgleichrechnung \ref{sec:met1} stark vom Realwert abweicht.
Dieser Wert ist in der ersten Zeile der Tabelle \ref{tab:fehler} zu finden.
Die durch Methode 2 und 3 berechneten Werte (zweite und dritte Spalte in Tab. \ref{tab:fehler}) haben eine ähnlich geringe Abweichung.
Wobei die relative Abweichung aller Fehler verglichen mit der Abweichung zum Realwert gering ist.\\
Die lineare Ausgleichsrechnung zeigt gerade bei niedrigen Frequenzen eine Abweichung zu den gemessenen Werten.
Die Abweichung ist dem Oszilloskop geschuldet, welches nur bis zu einer bestimmten Frequenz genaue Werte liefert.\\
Die Plots der nicht-linearen Ausgleichsrechnung stimmen mit der Theorie überein.
Wie theoretisch zu erwarten, konvergiert die Phasenverschiebung für hohe Frequenzen gegen $\varphi = \pi / 2$ (siehe Abb. \ref{fig:phase}).\\
Ebenso integriert der RC-Kreis alle angelegten Spannungen bei hohen Frequenzen (siehe Abb. \ref{tab:integration}).
\\
Zum Vergleich der Realwert der Zeitkonstante beträgt $ RC = \SI{1.027(56)}{\milli \second}$.
\begin{table}
\centering
\caption{Berechnete Werte im Vergleich zum Realwert.}
\begin{tabular}{c|ccc}
    \toprule
    Methode & Berechneter Wert $ \:/\: \si{\milli\second} $ & relativer Fehler\:/\: \% & Abweichung zum Realwert\:/\: \%\\
    \midrule
    \ref{sec:met1} & $RC = \SI{0.5969(229)}{}$ & 3.8364 & 41.88\\
    \ref{sec:met2} & $RC = \SI{1.2360(167)}{}$ & 1.3511 & 20.35\\
    \ref{sec:met3} & $RC = \SI{1.2685(214)}{}$ & 1.6870 & 23.51\\ 
    \bottomrule
\end{tabular}
\label{tab:fehler}
\end{table}
\FloatBarrier
Beachte man den Innenwiderstand $R_\text{i}=\SI{600}{\ohm}$ des Generators, so ergibt sich der neue Theoriewert
\begin{equation*}
    RC_\text{neu} = \SI{1.027(112)}{\milli\second}.
\end{equation*}
Dieser Wert liegt näher an den gemessenen Werten. Durch weitere Widerstände $R$ (z.B des Kabels) oder durch eine abweichende Kapazität $C$
kann der Unterschied zwischen Theorie- und Praxiswert begründet werden.