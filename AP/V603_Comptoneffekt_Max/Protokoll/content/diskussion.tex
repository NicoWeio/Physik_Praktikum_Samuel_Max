\section{Diskussion}
\label{sec:Diskussion}

Die bestimmten Energien der $K_\alpha$ und $K_\beta$ Linie sind
\begin{align*}
  E(K_\beta) & = \SI{1.4282e-15}{\joule}\\
  E(K_\alpha) &= \SI{1.2886e-15}{\joule}\\
\end{align*}
und weichen geringfügig von den Literaturwerte ab.
Die Literaturwerte wurden aus der Quelle \cite[6]{roentgen} entnommen.
Für die gesuchten Größen wurden dort die Werte
\begin{align*}
    E(K_\beta) &= \SI{1.4268e-15}{\joule}\\
    E(K_\alpha) &= \SI{ 1.2884e-15}{\joule}\\
\end{align*}
angegeben.
In der Quelle wurden nur die Wellenlängen angegeben, diese wurden mit der Gleichung \eqref{eq:energie} in die entsprechenden Energien umgerechnet.
Damit weicht die experimentelle Energie der $K_\beta$ Linie um $0.098\%$ vom Literaturwert ab.
Die Abweichung zwischen Literaturwert und gemessnen Wert der $K_\alpha$ Linie entspricht $0.015\%$.
Also konnte die Energie der $K_\alpha$ genauer bestimmt werden, als die der $K_\beta$ Linie.

Die berechnete Comptonwellenlänge des Elektrons entspricht
\begin{align*}
    \lambda_\text{C} &= \SI{3.91(7)}{\pico\meter}.\\
\end{align*}
Der Literaturwert der Comptonwellenlänge entspricht
\begin{align*}
    \lambda_\text{C} &= \SI{2.42}{\pico\meter}.\\
\end{align*}
Der Literaturwerte wurde aus der Quelle \cite{compton} entnommen.
Der berechnete Wert der Comptonwellenlänge weicht $\SI{80.429(2982)}{\percent}$ vom Literaturwert ab.
Die Abweichung könnte verringert werden, indem die Totzeitkorrektur oder die Integrationszeit des Messgerätes, besser in die Rechnung mit eingebracht werden würde. 