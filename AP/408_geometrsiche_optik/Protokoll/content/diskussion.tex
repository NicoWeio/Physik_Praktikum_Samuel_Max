\section{Diskussion}
\label{sec:Diskussion}
\subsection{Linsengleichung}
Die Brennweiten die bei der Verifizierung der Gleichung \eqref{eqn:linsengleichung} berechnet wurden wichen geringfügig von den reelen Brennweiten ab
\begin{align*}
 f_\text{reel} = \SI{15}{\centi\meter} \\
 \bar{f} = & \SI{14.384(37)}{\centi\meter}\\
 \Delta \bar{f} = \SI{0.62(4)}{\centi\meter}
\end{align*}
Dies ist damit zu befründen, dass bei einigen Messung der Schirm nicht genau im Brennpunkt stand oder die Linse und die Lichtquele nicht genau senkrecht zueinandern standen.
So kommt es auch bei der grafischen Methode zu geringen Abweichungen denn auch hier entprechen die berechneten Werte
\begin{align*}
    S_\text{x} = & \SI{14.83(11)}{\centi\meter}\\
    S_\text{y} = & \SI{14.09(7)}{\centi\meter}. \\
\end{align*}
nicht genau der Brennweite der Linse die verwendet wurde.
Sie weichen um die Werte 
\begin{align*}
   \Delta S_\text{x} = & \SI{0.17(11)}{\centi\meter} \\
    \Delta S_\text{y} = & \SI{0.91(7)}{\centi\meter}. \\
\end{align*}
von der richtigen Brennweite $f_\text{reel}$ ab.
Trotz der Abweichungen folgen die Geraden in der grafischen Methode aber den erwarteten Verläufen, die Abweichungen scheinen also gering zu sein.

\subsection{Bessel'sche Methode}
Bei der Messung der Brennweite durch die Bessel'sche Methode fällt auf, dass die Brennweiten von dem verwendeten Licht abhängen.
Dadurch ergeben sich die Mittelwerte 
\begin{align*}
    \bar{f_\text{weiß}} = & \SI{14.44(6)}{\centi\meter} \\
    \bar{f_\text{rot}}  = & \SI{14.58(5)}{\centi\meter} \\
    \bar{f_\text{blau}} = &  \SI{14.43(6)}{\centi\meter} \\
\end{align*}
der Brennweiten zu den jeweiligem Licht.
Dieses Phänomen ist damit zu erklären, dass Licht unterschiedlicher Wellenlänge unterschiedlich stark gebrochen wird.
So wird blaues Licht stärker als rotes Licht gebrochen und hat dementsprechend eine geringere Brennweite als rotes Licht.
Allerdings ist der Wert der Brennweite von blauem Licht und weißen Licht sehr ähnlich.
Dies könnte daran liegen, dass das verwendete weiße Licht einen hohen blau-Licht anteil besitzt.
Wahrscheinlicher ist allerdings, dass der Schirm bei der Messung des Brennwpunkts des weißen Lichts nicht genau im Brennpunkt sondern etwas zu nah an der Linse stand.

\subsection{Abbe'sche Methode}
Die bei der Auswertung erstellten Plots zur Abbe'schen Methode \ref{fig:abbe} folgen den erwarteten Verläufen und liegen so ziemlich genau auf einer Geraden.
Sie weisen dennoch kleinere Abweichungen von der Geraden auf, was zur folge hat, dass sich die beiden berechneten Brennweiten und Abstände zur Hauptebene
\begin{align*}
    f_{g'} = \SI{11.00(22)}{\centi\meter}, & \,\,h = \SI{0.3(4)}{\centi\meter}\\
    f_{b'} = \SI{13.12(8)}{\centi\meter}, & \,\, h = \SI{0.24(22)}{\centi\meter}
\end{align*}
geringfügig unterscheiden.
Auch hier ist der Grund für die Abweichungen wahrscheinlich, eine falsche Messung des Brennpunkts oder eine Fehlstelltung der Linsen.
