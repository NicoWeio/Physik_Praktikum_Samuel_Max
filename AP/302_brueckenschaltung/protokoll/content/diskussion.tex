\section{Diskussion}
\label{sec:Diskussion}
Die Unsicherheit des ohmschen Widerstands bei der Wheatstonschen Brücke ist mit einer Abweichung von $0,3\%$ ziemlich gering.
Im Vergleich dazu ist die Unsicherheit bei der Kapazitätsmessbrücke mit einer relativen Abweichung von $2,69\%$ für die Kapatizät und $22,43\%$ für den Widerstand sehr hoch.
Bei der Induktivitätsmessbrücke sind die relativen Abweichung mit $71,10\%$ für die Induktivität und $19,01\%$ für den Widerstand ebenfalls sehr hoch.
Besonders die Werte der Induktivität sind mit einer sehr hohen Unsicherheit belastet.
Dies ist damit zu Begründen, dass die komplexen Widerstände der Kondensatoren und Spulen schlecht zu messen sind.
Außerdem waren die Ausgleichswiderstände $R_2$ nicht variabel, wodurch der Phasenunterschied nicht kompensiert werden konnte, was zu weiteren Abweichungen führte.
\\\\
Die Unsicherheit der Mittelwerte für die Induktivität $L_x$ und den ohmschen Widerstand $R_x$, fallen bei der Maxwell-Brücke mit $8,62\%$ für den Widerstand und $23,81\%$ für die Induktivität wesentlich geringer als bei der Induktivitätsmessbrücke aus.
Das ist damit zu erklären, dass bei der Maxwell-Brücke, abgesehen vom Potentiometer für $R_3$ und $R_4$, keine variablen Widerstände eingebaut werden müssen.
\\\\
Bei den Messwerten die bei der Untersuchung der Wien-Robinson-Brücke aufgenommen wurden, fällt zunächst auf, dass die Werte im Bereich des Minimums nah an den theoretisch berechneten Werten liegen.
Das theoretische Minimum befindet sich bei der Frequenz, bei der auch das Minimum gemessen wurde.
Allerdings weichen alle Messwerte, die nicht in unmittelbarer Umgebung des Minimums $v_0$ liegen von der Theoriekurve ab.
Die Spannung in $v_0$ liegt nicht bei $\SI{0}{Volt}$, da der Generator einen gewissen Klirrfaktor hat und dadurch kein Minimum erreicht.
Durch die Annahme, dass die Summe aller Oberwellen nur aus einer Oberwelle besteht, ist der errechnete Klirrfaktor kleiner als der reale.