\section{Diskussion}
\label{sec:Diskussion}

\begin{table}
\centering
\caption{Vergleich der Messwerte mit Literaturwerten.}
\begin{tabular}{cccc}
    \toprule
    Stab & $\rho _\text{gemessen} \,/\, \si{\frac{\gram}{\cubic\centi\meter}}$ & $\rho _\text{referenz} \,/\, \si{\frac{\gram}{\cubic\centi\meter}}$  & Material \\
    \midrule
    eckig & $\SI{8.94(6)}{}$ & $\SI{8.90}{}$ & Nickel \\
    rund 1 & $\SI{8.410(28)}{}$ & $\SI{8.6}{}$ & Messing \\
    rund 2 & $\SI{2.808(6)}{}$ & $\SI{2.7}{}$& Aluminium \\
    \bottomrule
\end{tabular}
\label{tab:refferenz}
\end{table}

Wie in Tabelle \ref{tab:refferenz} zu sehen, besteht der eckige Stab aus Nickel, der einseitig eingespannte runde Stab aus Messing und der beidseitig eingespannte runde Stab aus Aluminium.
Die Referenzwerte für die Dichten wurden dabei auf der Website \cite{technik} gefunden.

\begin{table}
\centering
\caption{Vergleich der experimentell bestimmten Elastizitätsmodule mit Literaturwerten.}
\begin{tabular}{cccc}
    \toprule
    Stabart & $E_\text{gemessen}\,/\, \si{\giga\pascal}$ & $E_\text{referenz}$ & Abweichung $\,/\, \%$ \\
    \midrule
    $eckig_\text{einseitg}$ & $\SI{149.931(591)}{}$ & $214$ \cite{nickel} & 43.62\\
    $rund_\text{einseitg}$ & $ \SI{121.796(4012)}{}$ & $78-123$ \cite{messing} & 0 \\
    $eckig_\text{beidseitg}$ & $\SI{30(101)}{}$ & $214$ \cite{nickel}& 613.33 \\
    $rund_\text{beidseitig}$ & $\SI{80(4)}{}$ & $70$ \cite{nickel} & 12.5 \\
\end{tabular}
\label{tab:ele}
\end{table}

In der Tabelle \ref{tab:ele} werden die experimentell bestimmten Elastizitätsmodule mit Literaturwerten verglichen.
Zu sehen ist, dass besonders bei der beidseitigen Einspannung die Abweichung vom Literaturwert sehr groß ist.
Diese Abweichung vom Literaturwert ist beim beidseitig eingespannten eckigen Stab damit zu begründen, dass die Kraft die auf den Stab wirken konnte, durch die Anzahl an Gewichten, limitiert war.
Dadurch konnte der Stab nicht weit genug ausgelenkt werden um Messwerte zu gewinnen, die eine gute Auswertung garantieren.\\
Außerdem ist zu erkennen, dass die einseitige Einspannung allgemein bessere Messwerte liefert, da die Abweichungen dort geringer sind.
Aus den Schwierigkeiten bei der Messungen des eckigen Stabes aus Nickel lässt sich schließen, dass es leichter ist den Elastizitätsmodul von Stäben mit geringer Dichte zu bestimmen, da diese mit weniger Gewicht stärker Ausgelenkt werden können.