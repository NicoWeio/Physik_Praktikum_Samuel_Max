\section{Theorie}
\label{sec:Theorie}

Jeder Körper hat ein Trägheitsmoment $I$.
Dieses kann durch 
\begin{equation*}
    I = \sum_i r_i^2 m_i
\end{equation*}
beschrieben werden. Dabei steht $m_i$ für ein Massenelement, welches einen Abstand $r_i$ von der Drehachse hat.

Für infinitesimal kleine Massenelemente kann die Summe umgeschrieben werden in ein Integral.
Damit ergibt sich 
\begin{equation*}
    I = \int r^2 dm .
\end{equation*}

Wenn nun der Schwerpunkt des Körpers nicht auf der Drehachse liegt muss das Trägheitsmoment mit dem Satz von Steiner
\begin{equation}
    I = I_\text{S} + m a^2
    \label{eqn:steiner}
\end{equation}
berechnet werden. Hier steht $a$ für den Abstand des Schwerpunkts zur Drehachse und $I_\text{S}$ für das Trägheitsmoment im Schwerpunkt.

Bei Drehbewegungen, die durch eine Feder herbeigeführt werden, wirkt ein Drehmoment $M$.
Dieses setzt sich aus der Kraft $\vec{F}$ und den Abstand $\vec{r}$ zur Drehachse zusammen und sieht wie folgt aus
\begin{equation*}
    \vec{M} = \vec{F} \times \vec{r} .
\end{equation*} 

In diesem Versuch führt dieses Drehmoment zu einer Schwingung mit der Schwingungsdauer $T$ welche durch 
\begin{equation}
    T = 2\pi \sqrt{\frac{I}{D}}
    \label{eqn:dauer}
\end{equation}
beschrieben wird, diese gilt allerdings nur bei kleinen Auslenkungen.
Hierbei entspricht $D$ der Winkelrichtgröße der Feder.
Diese kann durch 
\begin{equation}
    D = \frac{M}{\phi}
    \label{eqn:federstarke}
\end{equation}
bestimmt werden, wobei $\phi$ die Auslenkung der Feder ist.

Nach Umformen von \ref{eqn:dauer} ergibt sich
\begin{equation}
    I = \frac{T^2 D}{4 \pi^2} .
    \label{eqn:tragheit}
\end{equation}

Im Versuch werden die Trägheitsmomente einiger einfacher geometrischer Körper bestimmt.
Das eines stehenden Zylinders mit Radius $R$ entspricht
\begin{equation}
    I_\text{zyl,s} = \frac{m R^2}{2}.
    \label{eqn:zyls}
\end{equation}
Das eines liegenden Zylinders mit Länge $h$ entspricht
\begin{equation}
    I_\text{zyl,l} = m \left ( \frac{R^2}{4} + \frac{h^2}{12} \right ).
    \label{eqn:zyll}
\end{equation}
Das einer Kugel mit Radius $R$ entspricht
\begin{equation}
    I_\text{K} = \frac{2}{5}mR^2 .
    \label{eqn:kugel}
\end{equation}
Das eines langen dünnen Stabes mit Länge $l$ entspricht
\begin{equation}
    I_\text{St} = \frac{ml^2}{12} .
    \label{eqn:stab}
\end{equation}

