\section{Diskussion}
\label{sec:Diskussion}
Die Winkelrichtgröße hat einen relativen Fehler von $\SI{43.75}{\%}$.
Bei weiteren Berechnungen werden die neuen Fehler durch die Fehlerfotpflanzung beschrieben.
Die Ausgleichsgerade zur Bestimmung des Eigenträgheitsmoment entspricht der Theorie und weicht wenig von den Messwerten ab.\\
Ein Teil der Abweichungen ist damit zu Begründen, dass die Messenung der Periodendauer schwierig war, da die Schwingungen in den meisten Fällen sehr kurz war.
Das Trägheitsmoment der Puppe mit ausgestreckten Beinen (Pos.1) ist wie zu erwarten größer als das der Puppe mit angelegten (Pos. 2).
\\
Die Abweichungen zwischen Theorie- und Praxiswert der Trägheitsmomente der Körper und der Modelpuppe sind konstant (Tab. \ref{tab:diff}) und lassen daher auf einen systematischen Fehler schließen.
Bei der Bestimmung des Eigenträgheitsmoment der Drillachse wird die Stange als masselos angenommen.
Daraus folgt dass das Trägheitsmoment der Stange in etwa
\begin{equation}
    I_\text{Abw} = \SI{0.00365}{\meter^2\kg}
\end{equation}
beträgt.
Abschließend kann gesagt werden, dass die erhaltenen Werte trotz ungenauer Messmethoden zur Theorie passen.
Außerdem darf die Masse der Stange nicht vernachlässigt werden, da sonst negative Trägheitsmomente auftreten.
\begin{table}
    \centering
    \begin{tabular}{c|ccc}
    \toprule
    Körper & $I_\text{th} / \si{\metre^2\kg}$ & $I_\text{exp} / \si{\metre^2\kg}$ & Differenz / \si{\metre^2\kg} \\
    \midrule
    Zylinder - Parallel &  $0,0008$ & $−0,0030$ & $0,0038$\\
    Zylinder - Senkrecht & $0,0005$ & $−0,0032$ & $0,0037$\\
    Modellpuppe - Pos. 1 & $0,000328$ & $−0,0032$ & $0,0035$\\
    Modellpuppe - Pos. 2 & $0,000136$ & $−0,0035$ & $0,0036$\\
    \bottomrule
    \end{tabular}
    \caption{Theoriewert, Praxiswert und Differenz der Trägheitsmomente verschiedener Körper.}
    \label{tab:diff}
\end{table}