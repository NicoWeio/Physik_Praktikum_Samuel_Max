\section{Diskussion}
\label{sec:Diskussion}

Die aufgenommenen Werte führen alle zu Diagrammen die den erwarteten Verläufen entsprechen.
Der experimentell bestimmte effektive Widerstand weicht $102.64 \%$ vom Theoriewert ab (siehe Tab. \ref{tab:fehler}).
Wird der Innenwiderstand des Generators $R_\text{i}=\SI{50}{\ohm}$ berücksichtigt, so ergibt sich eine Abweichung von $28.23 \%$.
\\
Der Widerstand bei dem aperiodischen Grenzfall $R_\text{ap}$ hat eine Abweichung von $12.59 \%$ (siehe Tab. \ref{tab:fehler}).
Der Übergang vom Kriechfall zum aperiodischen Grenzfall ist nicht direkt ablesbar.
Die Abweichung kommt zustande, da der aperiodische Grenzfall nicht genau vom Oszilloskop ablesbar ist.
\\
Für die Berechnung der Güte und der Breite der Resonanzkurve wurde der Innenwiderstand beachtet.
Der gemessene Wert der Güte $q$ zeigt eine Abweichung von $6.45 \%$ und entspricht ungefähr der Theorie.
\\
Die Abweichung der gemessenen Werte der Breite der Resonanzkurve $f_+ - f_-$ zum Theoriewert liegt bei $15.00 \%$ (siehe Tab. \ref{tab:fehler}).
\\

\begin{table}
\centering
\caption{Die relativen Abweichungen der Messwerte von den gegebene Werten.}
\begin{tabular}{cccc}
\toprule
 & Messwert & gegebener Wert & Abweichung $ \,/\, \%$ \\
\midrule
$R_1$ &  $\SI{136.176(403)}{\ohm}$ & $\SI{67.2(1)}{\ohm}$ & 102.64 \\
$R_\text{ap}$ & $\SI{5}{\kilo\ohm}$ & $\SI{5.72(4)}{\kilo\ohm}$ & 12.59 \\
$q$ & 3.657 & $\SI{3.909(029)}{}$ & 6.45 \\
$f_{+}- f_{-}$ & $\SI{8}{\kilo\hertz}$ & $\SI{6947(21)}{\hertz}$ & 15.00 \\
\bottomrule
\end{tabular}
\label{tab:fehler}
\end{table}