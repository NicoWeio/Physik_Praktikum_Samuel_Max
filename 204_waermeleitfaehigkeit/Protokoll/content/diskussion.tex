\section{Diskussion}
\label{sec:Diskussion}

%BalBlaBla
Die bestimmten Wärmeleitfähigkeiten 
\begin{align*}
\kappa _\text{messing} = \SI{80.763(6316)}{\frac{\watt}{\meter\kelvin}}     \\
\kappa _\text{aluminium} = \SI{197.071(20751)}{\frac{\watt}{\meter\kelvin}}\\
\kappa _\text{edelstahl} = \SI{14.418(879)}{\frac{\watt}{\meter\kelvin}}\\
\end{align*}
\FloatBarrier
weichen gerinfügig von den aus \cite{leitfaehigkeit} entnommen Literaturwerten ab.
Diese sind 
\begin{align*}
\kappa _\text{messing} = 81 - \SI{105}{\frac{\watt}{\meter\kelvin}}\\
\kappa _\text{aluminium} = \SI{220}{\frac{\watt}{\meter\kelvin}}\\
\kappa _\text{edelstahl} = \SI{20}{\frac{\watt}{\meter\kelvin}}\\
\end{align*}

Damit weicht $\kappa _\text{messing} 0.30 \%$, $\kappa _\text{aluminium} 11.63 \%$ und $\kappa _\text{edelstahl} 38\%$ vom Literaturwert ab.
Alle liegen dabei unter den Literaturwerten, was drauf schließen lässt, dass einige Wärme durch die unzureichend isolierte Messvorrichtung entwichen ist.
Außerdem musste während der dynamischen Messung die Spannung der Wärmequelle verringert werden, da sonst die Grundplatte zu heiß geworden wäre.
Dies führt zusätzlich zu Abweichungen zwischen Literaturwerten und Messwerten, da die Amplituden der Wärmewelle dann nicht mehr dem anfänglichem Muster folgen.
