\section{Theorie}
\label{sec:Theorie}
Besitzt ein Körper an veschiedenen Stellen unterschiedliche Temperaturen, so kommt es zu einem Wärmetransport vom wärmeren zum kälteren Gebiet.
Dieser Wärmetransport kann durch Konvektion, Wärmestrahlung oder Wärmeleitung geschehen.
Im Folgenden wird letzteres behandelt.\\
Die Wärme wird über Phononen und freibewegliche Elektronen transportiert.
Die transportierte Wärmemenge $\symup{d}Q$ in der Zeit $\symup{d}t$ eines Stabes mit recheckigem Querschnitt berrechnet sich nach
\begin{equation}
    \symup{d}Q = -\kappa A \frac{\partial T}{\partial x} \symup{d}t .
    \label{eqn:leit}
\end{equation}
Der Stab hat die Länge $L$, die Querschnittsfläche $A$ und das Material die Dichte $\rho$, die spezifische Wärme $c$ und die Wärmeleitfähigkeit $\kappa$.
Für die Wärmestromdichte gilt 
\begin{equation}
    j_\text{w} = -\kappa \frac{\partial T}{\partial x} .
\end{equation}
Die eindimensionale Wärmeleitungsgleichung
\begin{equation}
    \frac{\partial T}{\partial t} = \frac{\kappa}{\rho c} \frac{\partial^2 T}{\partial x^2}
    \label{eqn:wellengleichung}
\end{equation}
ergibt sich aus der Kontinuitätsgleichung.
Die Temperaturleitfähigkeit $\sigma_\text{T} = \frac{\kappa}{\rho c}$ gibt an, wie schnell sich der Temperaturunterschied ausgleicht.
Die Lösung der Diffusionsgleichung ist Abhängig von der Geometrie des Stabes und der Anfangsbedingungen.
\\
In einem Stab der mit der Periode $T$ abwechselnd erwärmt und abgekühlt wird, breitet sich eine Temperaturwelle der Form
\begin{equation}
    T(x,t) = T_\text{max} e^{-\sqrt{\frac{\omega \rho c}{2 \kappa}}x} \cdot \cos{\omega t - \sqrt{\frac{\omega \rho c}{2 \kappa}}x}
\end{equation}
aus.
Die Phasengeschwindigkeit der Welle beträgt
\begin{equation}
    v = \frac{\omega}{k} = \frac{2 \kappa \omega}{\rho c} .
\end{equation}
Die Wärmeleitfähigkeit ergibt sich aus dem Amplitudenverhältnis $\frac{A_\text{nah}}{A_\text{fern}}$ an zwei Messstellen $x_\text{nah}$, $x_\text{fern}$ und den Beziehungen $\omega = 2\pi / T$, $\phi = 2 \pi \Delta t / T$ mit der Periodendauer $T$ und der Winkelgeschwindigkeit $\omega$:
\begin{equation}
    \kappa = \frac{\rho c (\Delta x)^2}{2 \Delta t \frac{A_\text{nah}}{A_\text{fern}}}
    \label{eqn:kappa}
\end{equation}
Dabei ist $\Delta x$ der Abstand und $\Delta t$ die Phasendifferenz der Welle zwischen den Messstellen.