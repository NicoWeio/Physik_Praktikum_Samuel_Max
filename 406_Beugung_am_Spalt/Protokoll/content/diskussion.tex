\section{Diskussion}
\label{sec:Diskussion}

\subsection{Einzelspalt}
Die Funktion der Ausgleichsrechung stimmt sehr gut im Bereich des zentralen Maximums, mit den Messwerten und dem theoretischem Verlauf überein.
Allerdings gibt es besonders in den Bereichen niedriger Intensität Abweichungen von Theorie und Ausgleichs-Fit.
Es fällt außerdem auf, dass die Maximas des Ausgleichs-Fits und der Theoriekurve, abgesehen vom zentralen Maximum, leicht versetzt sind.
Dazu ist anzumerken, dass der Versuchsaufbau nicht den ausreichend stabil war und so schon kleine Bewegungen am Tisch oder den Geräten, dass Interferenzmuster verschoben haben.
Zudem war der Raum in dem das Experiment aufgebaut war zwar abgedunkelt, allerdings kam es dennoch zu Helligkeitsschwankungen, wenn zum Beispiel die Tür geöfnnet wurde oder ein Handy-Bildschirm angegangen ist.
Es war eine kleine Leselampe neben dem Versuchsaufbau, auch diese hat die gemessene Intensität besonders in den niedrigen Bereichen beeinflusst.
So weicht die berechnete Spaltenbreite $b_\text{exp}$ um $\Delta b = 0.0303\si{\milli\meter}$ von der tatsächlichen Spaltenbreite $b_\text{re}$ ab.
\begin{align*}
b_\text{exp} &= 0.1197 \si{\milli\meter}\\
b_\text{re} &= 0.15 \si{\milli\meter}
\end{align*}

\subsection{Doppelspalt}
Im Bereich des zentralen Maximums der Intensität, des Beugungsmusters am Doppelspalt kam es zu starker Streeung der Messwerte.
Dadurch liegt der Fit der Ausgleichsrechung nur bedingt in den Messwerten und stimmt nur an einigen Stellen mit dem theoretischem Verlauf überein.
Dennoch ist in den Messwerten ein zentrales Maximum gut zu erkennen und auch die Maximas direkt neben dem zentralen Maximum sind erkennbar.
Die Streuung der Messwerte im Bereich des Maximums ist mit wie beim Einzelspalt mit der unzureichenden Stabilität des Aufbaus zu begründen.
Es ist anzunehmen, dass durch leichte Stöße am Tisch oder dem Aufbau, der Laser während der Messung so verrutscht ist, dass dieser nur noch durch ein Spalt strahlte.
So ist auch die starke Abweichung von $\Delta b = 0.72515\si{\milli\meter}$ der experimentell bestimmten Spaltbreite $b_\text{exp}$ von der tatsächlichen Spaltbreite $b_\text{re}$ zu begründen.
\begin{align*}
b_\text{exp} &= 0.82515 \si{\milli\meter}\\
b_\text{re} &= 0.1 \si{\milli\meter}
\end{align*}
Zudem besitzt der berechnete Abstand zwischen den beiden Spalten $g_\text{exp} = -0.79287\si{\milli\meter}$ negativ, was definitiv nicht möglich ist.
Auch hier liegt die Erklärung nahe, dass der Laser nur durch eine der beiden Spalten strahlte und so das Interferenzmuster eines Einzelspaltes verursachte.
Der tatsächlich Wert des Spaltenabstand beträgt $g_\text{re} = 0.4\si{\milli\meter}$.
