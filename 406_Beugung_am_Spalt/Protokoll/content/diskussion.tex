\section{Diskussion}
\label{sec:Diskussion}

\subsection{Einzelspalt}
Die Funktion der Ausgleichsrechung im Bereich des zentralen Maximums stimmt sehr gut mit den Messwerten und dem theoretischem Verlauf überein.
Allerdings gibt es besonders in den Bereichen niedriger Intensitäten Abweichungen von der Theorie- und der Ausgleichskurve.
Zudem fällt auf, dass die Maxima des Ausgleichs-Fits und der Theoriekurve, abgesehen vom zentralen Maximum, leicht versetzt sind.
Dazu ist anzumerken, dass der Versuchsaufbau nicht ausreichend stabil war und so schon kleine Bewegungen am Tisch oder den Geräten, das Interferenzmuster verschoben haben könnten.
Zudem war der Raum in dem das Experiment aufgebaut war zwar abgedunkelt, allerdings kam es dennoch zu Helligkeitsschwankungen, wenn zum Beispiel die Tür geöfnnet wurde oder ein Handy-Bildschirm angegangen ist.
Es war eine kleine Leselampe neben dem Versuchsaufbau, auch diese hat die gemessene Intensität besonders in den niedrigen Bereichen beeinflusst.
So weicht die Spaltenbreite $b_\text{exp}$ um $\Delta b = \SI{0.0303}{\milli\meter}$ von der tatsächlichen Spaltenbreite $b_\text{re}$ ab.
\begin{align*}
b_\text{exp} &= \SI{0.1197}{\milli\meter}\\
b_\text{re} &= \SI{0.15}{\milli\meter}
\end{align*}

\subsection{Doppelspalt}
Im Bereich des zentralen Maximums der Intensität, des Beugungsmusters am Doppelspalt kam es zu starker Streuung der Messwerte.
Dadurch liegt der Fit der Ausgleichsrechung nur bedingt in den Messwerten und stimmt nur an einigen Stellen mit dem theoretischen Verlauf überein.
Dennoch ist in den Messwerten ein zentrales Maximum gut zu erkennen und auch die Maxima direkt neben dem zentralen Maximum sind erkennbar.
Die Streuung der Messwerte im Bereich des Maximums ist mit wie beim Einzelspalt mit der unzureichenden Stabilität des Aufbaus zu begründen.
Es ist anzunehmen, dass durch leichte Stöße am Tisch oder dem Aufbau, der Laser während der Messung verrutscht ist und so das Beugungsbild verfälscht wurde.
So ist auch die starke Abweichung von $\Delta b = 0.72515\si{\milli\meter}$ der experimentell bestimmten Spaltenbreite $b_\text{exp}$ von der tatsächlichen Spaltenbreite $b_\text{re}$ zu begründen.
\begin{align*}
b_\text{exp} &= \SI{0.82515}{\milli\meter}\\
b_\text{re} &= \SI{0.1}{\milli\meter}
\end{align*}
Zudem ist der berechnete Abstand zwischen den beiden Spalten $g_\text{exp} = -0.79287\si{\milli\meter}$ negativ, was physikalisch nicht möglich ist.
Ein möglicher Grund ist eine falsche Justierung des Lasers, sodass dieser nicht mit gleicher Intensität auf beide Spalten trifft.
Der tatsächliche Spaltenabstand beträgt $g_\text{re} = \SI{0.4}{\milli\meter}$.