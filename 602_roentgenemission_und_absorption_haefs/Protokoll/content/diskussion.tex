\section{Diskussion}
\label{sec:Diskussion}

%Bragg
Das experimentell bestimmte Maximum in Abschnitt \ref{sec:bragg} liegt bei $\theta_\text{exp} = \SI{28.2}{\degree}$.
Die Theorie erwartet das Maximum bei $\theta_\text{th} = \SI{28.0}{\degree}$.
Die Abweichung vom gemessenen zum theoretischen Wert beträgt $\Delta \theta = \SI{0.71}{\percent}$ und ist durch die grobe Messung in $\SI{0.1}{\degree}$-Schritten um den Peak zu begründen.
\\
%Emission
Die minimale Wellenlänge bzw. die maximale Energie des Bremsbergs kann nicht ermittelt werden, da die Messung in dem Bereich zu ungenau ist.
Für genauere Werte muss bis $\theta = \SI{19}{\degree}$ die Intensität in einer kleineren Schrittweite aufgenommen werden.
Für Abschnitt \ref{sec:emission} ergeben sich folgende Abweichungen zwischen Theorie- und Experimentalwert der Energien an der $K_\alpha$ und $K_\beta$-Linie: 
\begin{align*}
    \Delta E_{K,\alpha} = \SI{0.08}{\percent} \\
    \Delta E_{K,\beta} = \SI{0.21}{\percent}
\end{align*}
Die Abweichungen sind gering und können mit Messunsicherheiten der Geräte begründet werden.
\\
%Absorption
In Abschnitt \ref{sec:absorption} wurde die Absorptionsenergie, der Bragg-Winkel und die Abschirmkonstante experimentell bestimmt.
Die Tab. \ref{tab:vgl_elemente} zeigt die Abweichungen zwischen den Theorie- und experimentell bestimmten Werten.
\begin{table}
    \centering
    \begin{tabular}{c|ccc}
        \toprule
        Element & $\Delta \theta_K \,/\, \si{\percent}$ & $\Delta E_K \,/\, \si{\percent}$ & $\Delta \sigma_K \,/\, \si{\percent}$ \\
        \midrule
        Zink & 0.75 & 0.65 & 2.25 \\
        Gallium & 0.17 & 0.00 & 0.83 \\
        Brom & 0.08 & 0.00 & 0.00 \\
        Rubidium & 1.03 & 1.23 & 4.31 \\
        Strontium & 0.73 & 0.78 & 3.26 \\
        Zirkonium & 1.52 & 1.39 & 6.85 \\
        \bottomrule
    \end{tabular}
    \caption{Abweichung zwischen Experimental- und Theoriewert \cite{k_kante} für den Braggwinkel $\theta_K$, die Absorptionsenergie $E_K$ und die Abschirmkonstante $\sigma_K$.}
    \label{tab:vgl_elemente}
\end{table}
Die Experimentalwerte für Gallium, Brom und auch Zink stimmen mit den Theoriewerten weitgehens überein.
Die Abweichung für Elemente mit höherer Ordnungszahl ist relativ hoch.
Gerade die experimentell bestimmte Abschirmkonstante weicht bei diesen Elementen stark vom Theoriewert ab.
Die Sommerfeldsche Feinstrukturformel ist genauer für Elemente mit niedriger Ordnungszahl.
\\
Die durch eine Ausgleichsrechnung bestimmte Rydbergenergie beträgt
\begin{equation*}
    R_{\infty ,exp} = h \cdot R = \SI{1.981(30)e-18}{\joule} .
\end{equation*}
Der Literaturwert \cite{konst} liegt bei
\begin{equation*}
    R_{\infty ,th} = \SI{2.179871e-18}{\joule} .
\end{equation*}
Somit weicht die experimentell bestimmte Rydbergenergie um $\Delta R_\infty = \SI{9.12}{\percent}$ vom Theoriewert ab.
Es sind keine Fehler der Messgeräte oder andere statistische Fehler angegeben, daher können ungenaue Messergebnisse Grund der hohen Abweichung sein.
Das Geiger-Müller Zählrohr ist auch äußerer Strahlung ausgesetzt.
Daher können Impulse, welche von äußeren elektromagnetischen Wellen erzeugt wurden, gemessen worden sein.